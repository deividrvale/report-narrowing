%TODO: Make a revision in notation commands, they are very confusing...

% General
\newcommand{\signature}{\Sigma}             %The signature set.
\newcommand{\subs}{\mathcal{S}ub}           %The substitutions set.
\newcommand{\vars}[1]{\mathtt{vars}(#1)}   %The set of variables.
\newcommand{\dom}[1]{\mathtt{dom}(#1)}     %The domain.
\newcommand{\ran}[1]{\mathtt{ran}(#1)}      %The range of a substitution.
\newcommand{\vran}[1]{\mathtt{v}\ran{#1}}

% Nominal Terms and Equality
\newcommand{\atomSet}{\mathbb{A}}           %The set of all atoms.
\newcommand{\var}{\mathbb{X}}               %The set of all META variables.
\newcommand{\abs}[2]{\left[ #1 \right]#2}   %Abstraction term.
\newcommand{\support}[1]{\mathtt{supp}(#1)}          %Support of an permutation.
\newcommand{\permID}{\mathtt{id}}           %Identity permutation.
\newcommand{\pAction}[2]{#1 \cdot #2}       %The action of a permutation on a term.
\newcommand{\pGroup}{\mathbb{P}}            %The group of all permutations acting on the set fo atoms.
\newcommand{\swapping}[2]{(#1 \ \ #2)}      %The Swapping permutation.
\newcommand{\terms}{T(\signature,\atomSet,\var)}    %In context of nominal this is the set of terms.
\newcommand{\pos}[1]{\mathtt{pos}(#1)}

\newcommand{\narrow}{\mathbin{\rightsquigarrow}}
\newcommand{\trs}{\mathcal{R}}

\newcommand{\fresh}{\#}



\newcommand{\solution}[1]{\mathcal{S}(#1)} %solution set of the problem P
\newcommand{\ueq}[2]{#1\overset{?}{=}#2} %an unification equational problem

\newcommand{\eqTh}[1]{\mathbin{=_{#1}}} %Fist Order Equational Theory

\newcommand{\transRel}{\mathrel{\implies}}


%Notation for Equational Unification (general theory)

\newcommand{\sig}[1]{\mathcal{S}ig(#1)} %signature of a set of equations

% #1 The theory, #2 the problem set
\newcommand{\eqSolution}[2]{\mathcal{U}_{#1}(#2)}

%This defines an command with optional arguments
% The arguments are:
%m: mandatory - the theory set of equations name
%g: optional argument - CONVENTION: SEND 1 TO BE TRUE "?" means that the equation is an E-unfication problem
\DeclareDocumentCommand\eqUnif{ m g }{%
	\IfNoValueTF{#2}{=_{#1}}{\overset{?}{=}_{#1}}
}

%This defines an command with optional arguments
% The arguments are: #1 - the theory #2 - the variable set
\DeclareDocumentCommand\iqoless{ O{} O{} }{%
	\precsim_{#1}^{#2}
}
\DeclareDocumentCommand\iqogreater{ O{} O{} }{%
	\succsim_{#1}^{#2}
}
\DeclareDocumentCommand\iqoeq{ O{} O{} }{%
	\simeq_{#1}^{#2}
}

\DeclareDocumentCommand\iqoneq{ O{} O{} }{%
	\npreceq_{#1}^{#2}
}


\DeclareDocumentCommand\csu{ O{}}{%
	\mathcal{U}_{#1}
}
