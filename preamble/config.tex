% Configure the document math mode and theorems style.

%Fonts and Language
\usepackage[T1]{fontenc}
\usepackage[utf8]{inputenc}
\usepackage[english]{babel}

\usepackage{hyperref}

\usepackage[margin=1in]{geometry}

\usepackage{fancyhdr}
\pagestyle{fancy}

%load figures in full page
\usepackage{incgraph,tikz}

\usepackage[all]{xy}
\CompileMatrices

%Math Packages - General
\usepackage{amsmath,amsthm,amssymb,amsfonts,amscd, amsbsy,mathtools}
\usepackage{mathtools}
\usepackage{wasysym}

% ********************************** Tables ************************************
\usepackage{booktabs} % For professional looking tables
\usepackage{multirow}

%\usepackage{multicol}
%\usepackage{longtable}
%\usepackage{tabularx}

% Logic
\usepackage{bussproofs}

\everymath{\displaystyle}
\DeclareMathAlphabet{\mathcal}{OMS}{cmsy}{m}{n}

%Document Packages - General
\usepackage{enumerate}
\usepackage{faktor}
\usepackage{pdflscape}

%Add optional argument
\usepackage{xparse}

%Enviroments configuration
\theoremstyle{definition}\newtheorem{definition}{Definition}[section]
\newtheorem{example}{Example}[section]
\newtheorem{lemma}{Lemma}[section]
\newtheorem{theorem}{Theorem}[section]
\newtheorem{proposition}{Proposition}[section]
\newtheorem{corollary}{Corollary}[section]
\theoremstyle{remark}\newtheorem{remark}{Remark}[section]
