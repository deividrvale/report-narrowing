% !TeX root = ../main.tex
\section{Narrowing}

In this section, we discuss the use of narrowing techniques on solving equations over an equational theory $E$. By solving an equation we mean finding a substitution $\sigma$ such that $s \sigma \eqTh{E} t\sigma$. These solutions can be found by unification \textit{modulo} $E$ (if such a unification algorithm exists for this theory), but here we are interested in applications of rewriting theory to solving such equations.

We start the presentation by the so-called syntactic unification, i.e. $E$-unification with empty $E$. The transformation rules given below is due to Martelli and Montanari \cite{DBLP:books/el/RV01/BaaderS01}, the idea is to transform sets of equations to other sets of equations until a termination state is reached; that is, a solution state or a failure one.

\begin{definition}
	Let $\signature$ be a signature. An equational goal is a finite set of $\signature$-equations.
\end{definition}

\begin{table}[ht]
	\caption{Martelli-Montanari rules}
	\label{table:unify_inf_rules}
	{\small
		\hrule
		\vspace{10pt}

		\textbf{(1) Trivial}
        $$\{ \ueq{x}{x} \} \cup G \implies G $$

		\textbf{(2) Decompose}
		$$\{\ueq{f(s_1, \cdots, s_n)}{f(t_1, \dots, t_n)}\} \cup G \implies \{\ueq{s_1}{t_1}, \dots , \ueq{s_n}{t_n} \} \cup G$$

		\textbf{(3) Symbol Clash}
		$$\{ \ueq{f(s_1, \dots, s_n)}{g(t_1, \dots, t_n)} \} \cup G \implies \bot \; \text{if} \; f \neq g $$

		\textbf{(4) Orient}
		$$\{ \ueq{t}{x} \} \cup G \implies \{ \ueq{x}{t} \} \cup G \; \text{if} \; t \notin V$$

		\textbf{(5) Occurs Check}
		$$\{ \ueq{x}{t} \} \cup G \implies \bot \text{ se } x \in \vars{t} \text{ and } x \neq t$$

		\textbf{(6) Variable Elimination}
		$$\{ \ueq{x}{t} \} \cup G \implies G\{x \mapsto t\} \text{ if } x \notin \vars{t}$$
		\hrule
	}
\end{table}

The application of the above rules non-deterministically transforms goals into goals:
$$G_0 \transRel \cdots \transRel G_n$$
Each application of a rule will then called a \textit{elementary derivation step}. As for the case of (Variable Elimination), we may get some substitution on the way, we make them explicit by writing:
$$G_0 \transRel G_1 \transRel_{\sigma_1} G_2 \transRel_{\sigma_2} \cdots \transRel_{\sigma_i} \cdots \transRel_{\sigma_{n-1}} G_n$$
The computed solution of the derivation chain is then the composition of such substitutions in their order of appearance.

\begin{definition}
	A \textit{successful} derivation chain is a finite sequence of equational goals $G_0,G_1,\dots,G_n$ such that the last goal is empty. We also say a derivation chain has \textit{failed} if it last goal is the fail symbol $\bot$.
\end{definition}

\begin{example}\label{example:unif-examples}
	\begin{enumerate}
		\item We want to determine an mgu of the terms $f(g(x),h(x,u))$ and $f(z,h(f(y,y),z))$. That is, solving the equational goal:
		      \begin{align*}
			      \{ f(g(x),h(x,u)) & = f(z,h(f(y,y),z)) \}                                           \\
			                        & \overset{(2)}{\transRel} \{g(x) = z, h(x,u) = h(f(y,y),z)\}     \\
			                        & \overset{(4)}{\transRel} \{z = g(x), h(x,u) = h(f(y,y),z)\}     \\
			                        & \overset{(6)}{\transRel}_{[z/g(x)]} \{h(x,u) = h(f(y,y),g(x))\} \\
			                        & \overset{(2)}{\transRel} \{x = f(y,y), u = g(x) \}              \\
			                        & \overset{(6)}{\transRel}_{[x/f(y,y)]} \{ u = g(f(y,y)) \}       \\
			                        & \overset{(6)}{\transRel}_{[u/f(y,y)]} \emptyset
		      \end{align*}
		      We get as the computed solution to the problem the composition $[z/g(x)][x/f(y,y)][u/g(f(y,y))]$.

		\item A failing unification derivation:
		      \begin{align*}
			      \{ h(x,y,x) & = h(y,g(x),x)\}                                     \\
			                  & \overset{(2)}{\transRel} \{x = y, x = g(x), x = x\} \\
			                  & \overset{(5)}{\transRel} \bot
		      \end{align*}
	\end{enumerate}
\end{example}

It can be proved (see \cite{DBLP:books/el/RV01/BaaderS01}) that this set of rules derive a correct and terminating procedure for syntactic unification.

\begin{definition}\label{definition:narrow}
	Say a term $s$ \textit{narrows} to a term $t$ if there exists a non-variable position $p \in \pos{s}$, a variant $l \rightarrow r$ of a rewrite rule in $\trs$, and a substitution $\sigma$ satisfying two conditions:
	\begin{enumerate}
		\item $\sigma$ is a \textit{mgu} of $\restr{s}{p}$ and $l$,
		\item $t = \left( s[r]_{p} \right)\sigma$.
	\end{enumerate}
\end{definition}

The relation $\narrow$ is called \textit{narrowing relation}. We write $s \narrow_{\sigma}^{*} t$ if there exists a narrowing derivation
$$s = t_1 \narrow_{\sigma_1} t_2 \narrow_{\sigma_2} t_3 \narrow \cdots \narrow_{\sigma_{n-1}}t_n = t$$
and $\sigma$ is given by $\sigma := \sigma_1\sigma_2 \cdots \sigma_{n-1}$, the computed solution of the above narrowing derivation.

\begin{remark}
	Renaming of rewrite rules will be mandatory to ensure completeness of the narrowing approach. We always use a simple renaming such that $\vars{l} \cap \vars{s} = \emptyset$. This also extend to chains of narrowing derivations.
\end{remark}

In a rewrite step $s \contr_[p, l \contr r, \sigma] t$ we always assume that the applied rewrite rule has no variables in common with $s$ and $\sigma$ is restricted to variables occurring in $l$. Consequently, $\sigma$ is a most general unifier of $\restr{s}{p}$ and $l$, and $t = s[r\sigma]_p = (s[r]_p)\sigma$. Hence rewriting can be viewed as a special case of narrowing.

Narrowing was first introduced in the context of $E$-unification. Fay \cite{fay1978first} and Hullot \cite{hullot:cfunif} shows that narrowing is a complete method for solving equations in the theory defined by a confluent and terminating term rewriting system. In fact, narrowing derive from paramodulation (a first attempt for solving equations in arbitrary equational theories), but with the requirement that the theory can be represented as a convergent rewrite system. This approach also shows an important application for the Knuth-Bendix completion procedure: it prepares the way for solving equations over $E$, by delivering a complete TRS for $E$ (if possible).

For this set of technology we give the name \textit{narrowing}, we now present the first application (of solving equations modulo $E$) in the framework of transformation rules on sets of equational goals.

Consider an equational theory $E$ specified by a convergent rewrite system $\trs$, which is called the equational specification of $E$. The rules below extend the Martelli-Montanari unification on syntax terms to solve equations modulo $E$. \newpage
\begin{table}[ht]
	\caption{Narrowing rules}
	\label{table:narrowing_inf_rules}
	{\small
		\hrule
		\vspace{10pt}

		\textbf{(7) Left Narrowing} if $s \narrow_{\sigma} s'$
		$$\{ s = t \} \cup G \implies_{\sigma} \{ s' = t\sigma \} \cup G\sigma $$

		\textbf{(8) Right Narrowing} if $t \narrow_{\sigma} t'$
		$$\{ s = t \} \cup G \implies_{\sigma} \{ s\sigma = t' \} \cup G\sigma $$
		\hrule
	}
\end{table}

\begin{definition}
    A \textit{narrowing derivation} is a finite set of equational goals in which each step is obtained by using the rules (1)-(8). It is a successful narrowing derivation if it last goal is empty.
\end{definition}

\begin{example}
	Let $\trs = \{ g(a) \rightarrow a \}$ and consider the failing unification attempt of Example \ref{example:unif-examples}-(2). Observe that $h(y,g(x),x) \narrow_{[x / a]} h(y,a,a)$. Thus by the above rules:
	\begin{align*}
		\{ h(x,y,x) = h(y,g(x),x) \} & \overset{(8)}{\implies}_{[x/A]} \{ h(a,y,a) = h(y,a,a) \} \\
		                             & \overset{(2)}{\implies} \{ a = y, y = a, a = a \}         \\
		                             & \overset{(4,1)}{\implies} \{ y = a\}                      \\
		                             & \overset{(8)}{\implies}_{[x/A]} \{  a = a \}              \\
		                             & \overset{(1)}{\implies} \emptyset
	\end{align*}
	The equational problem is now a successful narrowing derivation with computed answer substitution $\sigma = [x / a, y / a]$.
\end{example}

In order to solve an equation $\ueq{s}{t}$ in an equational theory, corresponding to such a TRS, one can construct all possible narrowing derivations starting from the given equational goal until an equation $\ueq{s'}{t'}$ is obtained such that $s'$ and $t'$ are indeed syntactically unifiable. Note that, if this equational goal has a solution one always get a last equation of the form $\ueq{s}{s}$. We now investigate the semantic of solving equations using narrowing techniques.

If $(\signature, E)$ is an equational theory, write $[s = t]_E$ for the set of all solutions to the equation $s = t$ modulo $E$, i.e $\{\gamma \mid E \vdash s\gamma = t \gamma \}$. Moreover, if $X$ is some set of substitutions, let $\sigma X$ be the set $\{ \sigma \mid \gamma \in X \}$.

By definition, solutions of the equation $s \eqUnif{E}{?}t$ are indeed closed under any ``variable assignment'', that is, they are closed under applications of substitutions. As explained in \cite{10.1007/BFb0032044} -- Section 2.4; for every substitution $\sigma$ we have $[s = t]_E \supseteq \{ \sigma \tau \mid E \vdash s\sigma \tau = t \sigma \tau  \} = \sigma [s\sigma = t \sigma]_E$, there is in principle the possibility of a stepwise determination of $[s = t]_E$ by means of ``specializations'' of the solution set. It consists of two kind of steps. The first one is as jus described guess a component $\sigma$ of the solution set and narrows $[s = t ]_E$ to $\sigma[s = t]_E$. The other one is: apply one equation of $E$ in one of the sides of the equation $s = t$ in hand. It is clear that this two types of steps do not change the solution set of the equation $s \eqUnif{E}{?} t$. In this way, by interating steps of the first and second type, we can finally get an equation that is syntactically unifiable, as described above.

The narrowing relation was defined on terms rather equational goals. They act upon goals by the means of the above transformation rules.

We now turn our attention on completeness of narrowing. The propositions below will be useful to prove the various versions of ``Lifting Lemmas'' of this study.

\begin{proposition}\label{proposition:varst}
	If $t$ is a term and $\gamma$ a substitution then $\vars{t\gamma} = (\vars{t} \setminus \dom{\gamma}) \cup \vran{\restr{\gamma}{\vars{t}}}$.
\end{proposition}

\begin{proposition}\label{proposition:change-of-domain}
	Suppose we have substitutions $\gamma, \theta, \theta'$ and sets $A,B$ of variables such that $(B \setminus \dom{\gamma}) \cup \vran{\gamma} \subseteq A$. If $\theta =^{A} \theta'$ then $\gamma\theta =^{B} \gamma\theta'$.
\end{proposition}

\begin{proposition}\label{proposition:change-of-domain-normalized}
	Let $\trs$ be a TRS and suppose we have sets $A,B$ of variables and substitutions $\gamma, \theta, \theta'$ such that the following consditions are satisfied:
	\begin{enumerate}
		\item $\restr{\theta}{A}$ is $\trs$-normalized,
		\item $\theta =^A \gamma \theta'$,
		\item $B \subseteq (A \setminus \dom{\gamma}) \cup \vran{\restr{\gamma}{A}}$
	\end{enumerate}
	Then $\restr{\theta'}{B}$ is also normalized.
	\begin{proof}
		Let $x \in B$. We have to show that $x\theta'$ is an $\trs$-normal form. If $x \in A \setminus \dom{\gamma}$ then $x\theta' = x(\gamma\theta') = x\theta$ which is an $\trs$-normal form by the first condition. If $x \in \vran{\restr{\gamma}{A}}$ then there exists a variable $y \in A$ such that $x \in \vars{y\gamma}$. Also, by condition (2), we have $x\theta' \leq (y\gamma)\theta' = y\theta$. By condition (1) $y\theta$ is an $\trs$-normal form and hence its subterm $x\theta'$ is also an $\trs$-normal form.
	\end{proof}
\end{proposition}

\begin{lemma}[Lifting Lemma]\label{lemma:lifting-lemma-ordinary-narrowing}
	Let $\trs$ be a TRS. Suppose we have terms $s$ and $t$, a normalized substitution $\theta$ and a finite set of variables $V$ such that $\vars{s} \cup \dom{\theta} \subseteq V$ and $t = s\theta$. If $t \stc t'$ then there exist a term $s'$ and substitutions $\theta', \gamma$ such that:
	\begin{enumerate}
		\item $s \narrow^{*}_{\gamma} s'$,
		\item $t' = s'\theta'$,
		\item $\theta =^V \gamma\theta'$,
		\item $\theta'$ is $\trs$-normalized.
	\end{enumerate}
	Futhermore, we may assume that the narrowing derivation $s \narrow^{*}_{\sigma} s'$ and the rewrite sequence $t \stc t'$ employ the same rules at the same positions.

	\begin{proof}
		The proof is by induction on the length of the reduction sequence from $t$ to $t'$. If $t = t'$, a reduction of length zero, then the result crearly follows. Suppose $t \contr t_1 \stc t'$ is a reduction sequence of length $n+1$.

		\begin{figure}[h!]
			\begin{displaymath}
				\xymatrix{
                    s \ar@{~>}[r]_>>{\gamma_1} & s_1 \ar@{}[dr]|{I.H} \ar@{~>>}[r]_>>{\gamma'}^>>{*} & s' \\
                    t = s\theta \ar@{-->}[u] \ar@{->>}[r]_>>{\trs} & \ar@{-->}[u] t_1 \ar@{->>}[r]_>>{\trs}^>>{*} & t' = s'\theta' \ar@{-->}[u]
				}
			\end{displaymath}
			\caption{Lifting Lemma}
			\label{figure:lifting-lemma}
        \end{figure}

		Let $t$ contract to $t_1$ using a position $p \in \pos{t}$ and a variant $l \rightarrow r$ of a rule from $\trs$ such that $\vars{l} \cap V = \emptyset$. We use the fact that $t = s\theta$ to write $\restr{(s\theta)}{p} = l \tau$ for some substitution $\tau$ with $\dom{\tau} \subseteq \vars{l}$. Since $\theta$ is normalized we have $p$ is a non-variable position and hence $\restr{(s\theta)}{p} = (\restr{s}{p})\theta$. Let $\gamma = \tau \cup \theta$ so $(\restr{s}{p})\gamma = (\restr{s}{p})\theta = l \tau$ then $\restr{s}{p}$ and $l$ are unifiable. Consider $\gamma_1$ as an idempotent mgu of $\restr{s}{p}$ and $l$. By Lemma \ref{lemma:unifiers-preserve-variables} $\dom{\gamma_1} \cup \vran{\gamma_1} = \vars{\restr{s}{p}} \cup \vars{l}$. Let $s_1 = (s[r]_p)\gamma_1$. By definition \ref{definition:narrow},
		\begin{equation}\label{lifting-lemma:eqn:s-narrows-s_1}
			s \narrow_{\gamma_1} s_1
		\end{equation}

		Since $\gamma_1 \leq \gamma$, there exists a substitution $\rho$ such that $\gamma = \gamma_1 \rho$. Let $V_1 = (V \setminus \dom{\gamma_1}) \cup \vran{\gamma_1}$ and define $\theta_1 = \restr{\rho}{V_1}$. Clearly $\dom{\theta_1} \subseteq V_1$. Also,
		\begin{align*}
			\vars{s_1} & = \vars{s[r]_p \gamma_1}                                                       \\
			           & \subseteq \vars{s[l]_p \gamma_1}                                               \\
			           & = \vars{s \gamma_1} \text{, by Proposition \ref{proposition:change-of-domain}} \\
			           & \subseteq V_1.
		\end{align*}
		Therefore, $\vars{s_1} \cup \dom{\theta_1} \subseteq V_1$.

		Using the equality $\theta_1 =^{V_1} \rho$ we obtain
		\begin{align*}
			s_1 \theta_1 = s_1\rho & = ( (s[r]_p)\gamma_1)\rho \\
			                       & = (s[r]_p)\gamma          \\
			                       & = (s\gamma)[r\gamma]_p
		\end{align*}
		and since $V \cap \dom{\tau} = \emptyset$ (remember that $\vars{l} \cap V = \emptyset$ and $\dom{\tau} \subseteq \vars{l})$, one have $\gamma =^{V} \theta_1$. Likewise $\gamma =^{\vars{r}} \tau$. Hence the term $(s\gamma)[r\gamma]_p$ is equal to $(s\theta)[r\theta]_p = t_1$. Thus
		\begin{equation}\label{lifting-lemma:eqn:t_1-instance-s_1}
			t_1 = s_1\theta_1
		\end{equation}
		Next we show that $\gamma_1\theta_1 =^V \theta$. Proposition \ref{proposition:change-of-domain} yields $\gamma_1\theta_1 =^V \gamma_1\rho$. Since $\gamma =^V \theta$ and using the equality $\gamma = \gamma_1\rho$ we have
		\begin{align}
			\gamma_1 \theta_1 & =^V \gamma_1 \rho \nonumber                            \\
			                  & =^V \gamma \nonumber                                   \\
			                  & =^V \label{lifting-lemma:gamma1-instance-theta} \theta
		\end{align}

		Finally we show that $\theta_1$ is normalized. Since $\dom{\theta_1} \subseteq V_1$ if suffices to show that $\restr{\theta_1}{V_1}$ is normalized. Let $B = (V \setminus \dom{\gamma_1}) \cup \vars{\restr{\gamma_1}{V}}$. Proposition \ref{proposition:change-of-domain-normalized} (with $A = V$) yields the normalization of $\restr{\theta_1}{B}$. Remember that $\vran{\gamma_1} \subseteq \vars{\restr{s}{p}} \cup \vars{l}$. Let $x \in \vran{\gamma_1}$. Idempotence of $\gamma_1$ yields $x \notin \dom{\gamma_1}$. If $x \in \vars{\restr{s}{p}} \subseteq V$ then $x \in V \setminus \dom{\gamma_1}$. If $x \in \vars{l}$ then $x \in \vars{l\gamma_1} = \vars{(\restr{s}{p})\gamma_1}$ and thus $x \in \vran{\restr{\gamma_1}{V}}$. So $\vran{\gamma_1} \subseteq B$, and hence $B = V_1$. So $\theta_1$ is normalized.

		The induction hypothesis give us a term $s'$ and substitutions $\theta'$, $\gamma'$ such that
		\begin{align}
			 & s_1 \narrow_{\gamma'}^{*} s' \label{lifting-lemma:narrow:s_1-narrows-to-s-prime}       \\
			 & t' = s'\theta'                                                                         \\
			 & \theta_1 =^{V_1} \gamma'\theta' \label{lifting-lemma:eqn:theta_1-instance-theta-prime} \\
			 & \theta' \text{ is normalized} \label{lifting-lemma:eqn:theta-prime-is-normalized}
		\end{align}
		Moreover, we can assume that $s_1 \narrow_{\gamma'}^{*} s'$ and $t_1 \stc t'$ using the same rules at the same positions. Let $\gamma = \gamma_1 \gamma'$. By joining (\ref{lifting-lemma:eqn:s-narrows-s_1}) and (\ref{lifting-lemma:narrow:s_1-narrows-to-s-prime}) we get $s \narrow_{\gamma}^{*} s'$. By construction this narrowing derivation employs the same positions as the rewrite sequence $t \stc t'$. It remains to show that $\gamma \theta' =^V \theta$. Proposition \ref{proposition:change-of-domain} applied to (\ref{lifting-lemma:eqn:theta_1-instance-theta-prime}) give $\gamma_1 \gamma'\theta' =^V \gamma_1\theta_1$ and hence
		\begin{align*}
			\gamma \theta' & =^V \gamma_1\theta_1                                                       \\
			               & = \theta, \text{ by equation (\ref{lifting-lemma:gamma1-instance-theta}).}
		\end{align*}
	\end{proof}
\end{lemma}

Now we have the following theorem which express the completeness of narrowing on solving equations modulo $E$.
\begin{theorem}[Narrowing Completeness]\label{theorem:narrowing-completeness}
    Let $\trs$ be a complete TRS for the equational theory $E$. Let, moreover, $s, t$ be terms and $\sigma \in [s = t]_E$. Then there is a successful derivation starting with $P_0 = \{ s = t \}$, using the rules (1)-(8), such that the computed solution $\tau$ is a solution for $s = t$ with $\tau \leq \sigma$.

    \begin{proof}
        Let $\trsNF{\sigma}$ be the $\trs$-normal form of $\sigma$. Notice that $\sigma =_{E} \trsNF{\sigma}$, hence $s \trsNF{\sigma} =_{E} t \trsNF{\sigma}$. Confluence of $\trs$ yields the existence of a common reduct $r$, that is, $s\trsNF{\sigma} \stc r \stcR t \trsNF{\sigma} $. Now using the Lifting Lemma with $V_1 = \vars{s} \cup \vars{t}$ we get a term $s'$ and substitutions $\rho, \tau$ such that
        \begin{enumerate}
            \item $s \narrow_\tau^* s'$
            \item $r = s' \rho$,
            \item $\tau \rho =^{V_1} \trsNF{\sigma}$
        \end{enumerate}
        Hence we have a derivation $\{s = t \} \implies \cdots \implies \{s' = t\tau\}$ consisting of left narrowing steps with computed answer substitution $\tau$.

        In a similar way, using (3) above, it follows that $t \tau \rho = t \trsNF{\sigma} \sigma \stc r$. We again apply the Lifting Lemma (with $V_2 = \vars{s',s,t}$) which give us a term $t'$ and substitutions $\gamma, \eta$ such that
        \begin{enumerate}
            \setcounter{enumi}{3}
            \item $t \tau \narrow_{\eta}^{*} t'$,
            \item $r = t' \gamma$,
            \item $\eta \gamma =^{V_2} \rho$.
        \end{enumerate}
        Again we can prolong the above narrowing derivation with a sequence of right narrowing steps
        $$\{s' = t\} \implies \cdots \{s'\gamma = t'\}$$
        and obtain as compute answer $\tau \eta$.

        Using (6), (5) and (2) above one get
        $$s'\eta\gamma = s'\rho = r = t'\gamma$$
        So $s'\eta$ and $t'$ are unifiable. The Martelli-Montanari rules yields an mgu $\gamma'$ of $s'\eta$ and $t'$. So one have a successful derivation with computed solution $\tau \eta \gamma'$. This is indeed a solution: since $s \narrow_{\tau}^* s'$ and $t\tau \narrow_{\eta}^{*} t'$, we have $s\tau =_E s'$ and
        $$t\tau \eta \gamma' =_E s'\eta\gamma' = t'\gamma' =_E t\tau\eta\gamma'.$$
        It remains to show that $\tau\eta \gamma' \iqoless[E][V_1]$. Since $\gamma'$ is most general, there exists a $\rho'$ such that
        $$\tau \eta \gamma' \rho' = \tau \eta \gamma =_E \tau \rho =_E \trsNF{\sigma} =_E \sigma.$$
        Hence, $\tau \eta \gamma' \iqoless[E][V_1] \sigma$. As required.
    \end{proof}
    From the above proof one can see that we can indeed drop the strong normalization condition if only normalized substitution is considered. Strong normalization of $\trs$ is only used in the normalization of $\sigma$ to $\trsNF{\sigma}$, hence we can strengthen this result by dropping the strong normalization requirement and restricting ourselves to normalizable substitutions. Since in a weakly normalizing TRS every substitution is normalizable, we obtain the following result.
    \begin{corollary}
        Narrowing is complete for weakly normalizing and confluent Term Rewriting Systems.
    \end{corollary}
\end{theorem}

Now with the completeness result in hand, we can think of matters of efficiency. The first problem one will face when using narrowing techniques is its large search space. The search for possible derivations can even be an infinite set (see Example below). In order to reduce the search space of narrowing, Hullot \cite{hullot:cfunif} introduced the concept of basic narrowing. He showed that basic narrowing is complete for convergent TRSs. Motivated by the example below we study, in the next section, a restricted form of narrowing called basic narrowing.

\begin{example}\label{example:app-concat-standard-narrowing}
	Consider the TRS
	\[
		\trs =
		\begin{cases}
			\rho_1: \appFunc{\nilList}{x} \contr x \\
			\rho_2: \appFunc{x \cdot y}{z} \contr x \cdot \appFunc{y}{z}
		\end{cases}
    \]
    which represents the application function over $\cdot$ (conc function on lists). By a simple analysis we can see that $\trs$ is convergent.

    \begin{landscape}
        \thispagestyle{empty}
        \hrule
        \begin{figure}[h!]
            \begin{displaymath}
                \xymatrix{
                     & \{ \appFunc{\appFunc{x}{y}}{z} = \nilList \} \ar@{=>}[dl]_>>{\sigma_1}_a \ar@{=>}[dr]^>>{\gamma_1}^b & & \\
                    \{ \appFunc{x_1}{z} = \nilList \} \ar@{=>}[d]_>>{\sigma_2}  & & \{ \appFunc{w_1 \cdot \appFunc{y_1}{u_1}}{z} = \nilList \} \ar@{=>}[dl]_>>{\gamma_2}_c \ar@{=>}[dr]^>>{\gamma_3}^d & \\
                    \{ x_2 = \nilList \} \ar@{=>}[d]_>>{\sigma_3} & \{ w_2 \cdot \appFunc{\appFunc{y_1}{u_1}}{u_2} = \nilList \} \ar@{=>}[d] & & \{ \appFunc{w_1 \cdot x_5}{z} = \nilList \} \ar@{=>}[d]_>>{\gamma_4}^e \\
                    \{ \nilList = \nilList \} \ar@{=>}[d] & \vdots & & \{ w_1 \cdot \appFunc{x_5}{z} \} \ar@{=>}[d] \\
                    \emptyset & & &  \vdots \\
                }
            \end{displaymath}
            \caption{Derivation tree for the objective $\{ \appFunc{\appFunc{x}{y}}{z} = \nilList \}$}
            \label{figure:example:app:derivation-tree}
        \end{figure}
        \hrule
        \begin{table}[h!]
            \caption{Narrowing steps}
            \label{table:nice_table}
            \centering
            \begin{tabular}{l  c  c}
            \cline{1-3}
                step & rewrite rule & narrowing substitution \\
            \hline
            a & $\appFunc{\nilList}{x_1} \contr x_1$ & $[x / \nilList, y / x_1]$ \\

            b & $\appFunc{w_1 \cdot v_1}{u_1} \contr w_1 \cdot \appFunc{v_1}{u_1}$ & $[x / w_1 \cdot v_1, y / u_1 ]$ \\

            c & $\appFunc{w_2 \cdot v_2}{u_2} \contr w_2 \cdot \appFunc{v_2}{u_2}$ & $[w_1 / w_2, v_2 / \appFunc{v_1}{u_1}, z / u_2]$ \\

            d & $\appFunc{\nilList}{x_5} \contr x_5$ & $[v_1 / \nilList, u_1 / x_5]$ \\

            e & $\appFunc{w_3 \cdot v_3}{u_3} \contr w_3 \cdot \appFunc{v_3}{u_3}$ & $[w_3 / w_1, v_3 / x_5, u_3 / z]$ \\
            \hline
            \end{tabular}
        \end{table}
    \end{landscape}
\end{example}



