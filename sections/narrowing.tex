% !TeX root = ../main.tex
\section{Narrowing}

In this section, we discuss the use of narrowing techniques on solving equations over an equational theory $E$. By solving an equation we mean finding a substitution $\sigma$ such that $s \sigma \eqTh{E} t\sigma$. These solutions can be found by unification \textit{modulo} $E$ (if such a unification algorithm exists for this theory), but here we are interested in applications of rewriting theory to solving such equations.

We start the presentation by the so-called syntactic unification, i.e. $E$-unification with empty $E$. The transformation rules given below is due to Martelli and Montanari [citation], the idea is to transform sets of equations to other sets of equations until a termination state is reached; that is, a solution state or a failure one. We also extend this same set of rules for solving $E$-equations.

\begin{definition}
	Let $\signature$ be a signature. An equational goal is a finite set of $\signature$-equations.
\end{definition}

\begin{table}[ht]
	\caption{Martelli-Montanari rules}
	\label{table:unify_inf_rules}
	{\small
		\hrule
		\vspace{10pt}

		\textbf{(1) Trivial}
		$$\{ \ueq{x}{x} \} \cup G \implies G $$
		Delete trivial equations.

		\textbf{(2) Decompose}
		$$\{\ueq{f(s_1, \cdots, s_n)}{f(t_1, \dots, t_n)}\} \cup G \implies \{\ueq{s_1}{t_1}, \dots , \ueq{s_n}{t_n} \} \cup G$$

		\textbf{(3) Symbol Clash}
		$$\{ \ueq{f(s_1, \dots, s_n)}{g(t_1, \dots, t_n)} \} \cup G \implies \bot \; \text{if} \; f \neq g $$

		\textbf{(4) Orient}
		$$\{ \ueq{t}{x} \} \cup G \implies \{ \ueq{x}{t} \} \cup G \; \text{if} \; t \notin V$$

		\textbf{(5) Occurs Check}
		$$\{ \ueq{x}{t} \} \cup G \implies \bot \text{ se } x \in \vars{t} \text{ and } x \neq t$$

		\textbf{(6) Variable Elimination}
		$$\{ \ueq{x}{t} \} \cup G \implies G\{x \mapsto t\} \text{ if } x \notin \vars{t}$$
		\hrule
	}
\end{table}

The application of the above rules non-deterministically transforms goals into goals:
$$G_0 \transRel \cdots \transRel G_n$$
Each application of a rule will then called a \textit{elementary derivation step}. As for the case of (Variable Elimination), we may get some substitution on the way, we make them explicit by writing:
$$G_0 \transRel G_1 \transRel_{\sigma_1} G_2 \transRel_{\sigma_2} \cdots \transRel_{\sigma_i} \cdots \transRel_{\sigma_{n-1}} G_n$$
The computed solution of the derivation chain is then the composition of such substitutions in their order of appearance.

\begin{definition}
	A \textit{successful} derivation chain is a finite sequence of equational goals $G_0,G_1,\dots,G_n$ such that the last goal is empty. We also say a derivation chain has \textit{failed} if it end is the fail symbol $\bot$.
\end{definition}

\begin{example}\label{example:unif-examples}
	\begin{enumerate}
		\item We want to determine an mgu of the terms $f(g(x),h(x,u))$ and $f(z,h(f(y,y),z))$. That is, solving the equational goal:
		      \begin{align*}
			      \{ f(g(x),h(x,u)) & = f(z,h(f(y,y),z)) \}                                           \\
			                        & \overset{(2)}{\transRel} \{g(x) = z, h(x,u) = h(f(y,y),z)\}     \\
			                        & \overset{(4)}{\transRel} \{z = g(x), h(x,u) = h(f(y,y),z)\}     \\
			                        & \overset{(6)}{\transRel}_{[z/g(x)]} \{h(x,u) = h(f(y,y),g(x))\} \\
			                        & \overset{(2)}{\transRel} \{x = f(y,y), u = g(x) \}              \\
			                        & \overset{(6)}{\transRel}_{[x/f(y,y)]} \{ u = g(f(y,y)) \}       \\
			                        & \overset{(6)}{\transRel}_{[u/f(y,y)]} \emptyset
		      \end{align*}
		      We get as the computed solution to the problem the composition $[z/g(x)][x/f(y,y)][u/g(f(y,y))]$.

		\item A failing unification derivation:
		      \begin{align*}
			      \{ h(x,y,x) & = h(y,g(x),x)\}                                     \\
			                  & \overset{(2)}{\transRel} \{x = y, x = g(x), x = x\} \\
			                  & \overset{(5)}{\transRel} \bot
		      \end{align*}
	\end{enumerate}
\end{example}

It can be proved that this set of rules derive a correct and terminating procedure for syntactic unification. (add citation here).

The situation changes if we want to solve equations for a non-empty equation theory $E$. This case is studied in the next section.

\section{Narrowing}

\begin{definition}
	Say a term $s$ \textit{narrows} to a term $t$ if there exists a non-variable position $p \in \pos{s}$, a variant $l \rightarrow r$ of a rewrite rule in $\trs$, and a substitution $\sigma$ satisfying two conditions:
	\begin{enumerate}
		\item $\sigma$ is a \textit{mgu} of $s|_p$ and $l$,
		\item $t = \left( s[r]_{p} \right)\sigma$.
	\end{enumerate}
\end{definition}

The relation $\narrow$ is called \textit{narrowing relation}. We write $s \narrow_{\sigma}^{*} t$ if there exists a narrowing derivation
$$s = t_1 \narrow_{\sigma_1} t_2 \narrow_{\sigma_2} t_3 \narrow \cdots \narrow_{\sigma_{n-1}}t_n = t$$
and $\sigma$ is given by $\sigma := \sigma_1\sigma_2 \cdots \sigma_{n-1}$. We consider $\sigma$ as the computed solution to the above narrowing derivation.

\begin{remark}
	Renaming of rewrite rules will be mandatory to ensure completeness of the narrowing approach. We always use a simple renaming such that $\vars{l} \cap \vars{s} = \emptyset$. This also extend to chains of narrowing derivations.
\end{remark}

Narrowing was first introduced in the context of $E$-unification. Fay [cite] and Hullot [23] shows that narrowing is a complete method for solving equations in the theory defined by a confluent and terminating term rewriting system. In fact, narrowing is a first attempt for solving equations in arbitrary equational theories $E$, with the requirement that they can be represented as a convergent rewrite system. This approach also shows an important application for the Knuth-Bendix completion procedure: it prepares the way for solving equations over $E$, by delivering a complete TRS for $E$ (if possible).

For this set of technology we give the name narrowing, we now present the first application (of solving equations modulo $E$) in the framework of transformation rules on sets of equational goals.

Consider an equational theory $E$ specified by a convergent rewrite system $\trs$, which is called the equational specification of $E$. \newpage
\begin{table}[ht]
	\caption{Narrowing rules}
	\label{table:narrowing_inf_rules}
	{\small
		\hrule
		\vspace{10pt}

		\textbf{(7) Left Narrowing} if $s \narrow_{\sigma} s'$
		$$\{ s = t \} \cup G \implies_{\sigma} \{ s' = t\sigma \} \cup G\sigma $$

		\textbf{(8) Right Narrowing} if $t \narrow_{\sigma} t'$
		$$\{ s = t \} \cup G \implies_{\sigma} \{ s\sigma = t' \} \cup G\sigma $$
		\hrule
	}
\end{table}

\begin{example}
	Let $\trs = \{ g(a) \rightarrow a \}$ and consider the failing unification attempt of Example \ref{example:unif-examples}-(2). Observe that $h(y,g(x),x) \narrow_{[x / a]} h(y,a,a)$. Thus by the above rules:
	\begin{align*}
		\{ h(x,y,x) = h(y,g(x),x) \} & \overset{(8)}{\implies}_{[x/A]} \{ h(a,y,a) = h(y,a,a) \} \\
		                             & \overset{(2)}{\implies} \{ a = y, y = a, a = a \}         \\
		                             & \overset{(4,1)}{\implies} \{ y = a\}                      \\
		                             & \overset{(8)}{\implies}_{[x/A]} \{  a = a \}              \\
		                             & \overset{(1)}{\implies} \emptyset
	\end{align*}
	The equational problem is now a successful narrowing derivation with computed answer substitution $\sigma = [x / a, y / a]$.
\end{example}

In order to solve an equation $s = t$ in an equational theory, corresponding to such a TRS, one can construct all possible narrowing derivations starting from the given equational goal until an equation $s' = t'$ is obtained such that $s'$ and $t'$ are indeed syntactically unifiable. Note that, if this equational goal has a solution we always will get a last equation of the form $s = s$. We now investigate the semantic of solving equations using narrowing techniques.

If $(\signature, E)$ is an equational theory, write $[s = t]_E$ for the set of all solutions to the equation $s = t$ modulo $E$. Moreover, if $X$ is some set of substitutions, let $X \sigma$ be the set $\{ \gamma \sigma \mid \gamma \in X \}$.

The narrowing relation was defined on terms rather equational goals. They act upon goals by the means of the above transformation rules.

\begin{example}
	Consider the TRS
	\[
		\trs =
		\begin{cases}
			\rho_1: 0 + x \rightarrow x \\
			\rho_2: s(x) + y \rightarrow s(x + y)
		\end{cases}
	\]
\end{example}

\begin{displaymath}
    \xymatrix{
        A \ar@{~>>}[r]_{\sigma}^{*} & B \ar[d] \\
        t = s\theta \ar[r] & C
    }
\end{displaymath}
