% !TeX root = ../main.tex
\section{Narrowing}

In this section, we discuss the use of narrowing techniques on solving equations over an equational theory $E$. By solving an equation we mean finding a substitution $\sigma$ such that $s \sigma \eqTh{E} t\sigma$. These solutions can be found by unification \textit{modulo} $E$ (if such unification algorithm exists for this theory), but here we are interested in applications of rewriting theory to solving such equations.

We start the presentation by the so-called syntactic unification, i.e. $E$-unification with empty $E$. The transformation rules given below is due to Martelli and Montanari [citation], the idea is to transform sets of equations to other sets of equations until a termination state is reached; that is, a solution state or a failure one. We also extend this same set of rules to solving $E$-equations.

\begin{definition}
    Let $\signature$ be a signature. An equational goal is a finite set of $\signature$-equations.
\end{definition}

\begin{table}[ht]
	\caption{Martelli-Montanari rules}
	\label{table:unify_inf_rules}
	{\small
		\hrule
		\vspace{10pt}

		\textbf{(1) Trivial}
		$$\{ \ueq{x}{x} \} \cup G \implies G $$
		Delete trivial equations.

		\textbf{(2) Decompose}
        $$\{\ueq{f(s_1, \cdots, s_n)}{f(t_1, \dots, t_n)}\} \cup G \implies \{\ueq{s_1}{t_1}, \dots , \ueq{s_n}{t_n} \} \cup G$$

		\textbf{(3) Symbol Clash}
		$$\{ \ueq{f(s_1, \dots, s_n)}{g(t_1, \dots, t_n)} \} \cup G \implies \bot \; \text{if} \; f \neq g $$

		\textbf{(4) Orient}
		$$\{ \ueq{t}{x} \} \cup G \implies \{ \ueq{x}{t} \} \cup G \; \text{if} \; t \notin V$$

		\textbf{(5) Occurs Check}
		$$\{ \ueq{x}{t} \} \cup G \implies \bot \text{ se } x \in \vars{t} \text{ and } x \neq t$$

		\textbf{(6) Variable Elimination}
		$$\{ \ueq{x}{t} \} \cup G \implies G\{x \mapsto t\} \text{ if } x \notin \vars{t}$$
		\hrule
	}
\end{table}

The application of the above rules non-deterministically transforms goals into goals:
$$G_0 \transRel \cdots \transRel G_n$$
Each application of a rule will then called a \textit{elementary derivation step}. As for the case of (Variable Elimination), we may get some substitution on the way, we make them explicit by writing:
$$G_0 \transRel G_1 \transRel_{\sigma_1} G_2 \transRel_{\sigma_2} \cdots \transRel_{\sigma_i} \cdots \transRel_{\sigma_{n-1}} G_n$$
The computed solution of the derivation chain is then the composition of such substitutions in their order of appearance.

\begin{definition}
    A \textit{successful} derivation chain is a finite sequence of equational goals $G_0,G_1,\dots,G_n$ such that the last goal is empty. We also say a derivation chain has \textit{failed} if it end is the fail symbol $\bot$.
\end{definition}

\begin{example}
    \begin{enumerate}
        \item We want to determine an mgu of the terms $f(g(x),h(x,u))$ and $f(z,h(f(y,y),z))$. That is, solving the equational goal:
        \begin{align*}
            \{ f(g(x),h(x,u)) &= f(z,h(f(y,y),z)) \}\\
            &\overset{(2)}{\transRel} \{g(x) = z, h(x,u) = h(f(y,y),z)\} \\
            &\overset{(4)}{\transRel} \{z = g(x), h(x,u) = h(f(y,y),z)\} \\
            &\overset{(6)}{\transRel}_{[z/g(x)]} \{h(x,u) = h(f(y,y),g(x))\} \\
            &\overset{(2)}{\transRel} \{x = f(y,y), u = g(x) \} \\
            &\overset{(6)}{\transRel}_{[x/f(y,y)]} \{ u = g(f(y,y)) \} \\
            &\overset{(6)}{\transRel}_{[u/f(y,y)]} \emptyset
        \end{align*}
        We get as the computed solution to the problem the composition $[z/g(x)][x/f(y,y)][u/g(f(y,y))]$.

        \item A failing unification derivation:
        \begin{align*}
            \{ h(x,y,x) &= h(y,g(x),x)\}\\
            &\overset{(2)}{\transRel} \{x = y, x = g(x), x = x\} \\
            &\overset{(5)}{\transRel} \bot
        \end{align*}
    \end{enumerate}
\end{example}

It can be proved that this set of rules derive a correct and terminating procedure for syntactic unification. (add citation here).

The situation changes if we whant to solve equations for a non-empty equation theory $E$.
