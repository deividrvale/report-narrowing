% !TeX root = ../main.tex
\section{Preliminaries}
In this section we review the essential notions used in this work. The notation is consistent with major works on unification and rewriting. The reader is referred to \cite{bezem2003term} for a more detaled discussion on rewriting.

A signature is a set $\signature$ of function symbols. Associated with every $f \in \signature$ there is a non-negative integer $n$, the arity of the function symbol $f$. We denote this by writing $f:n$. Function symbols of arity $0$ is called \textit{constants}. Futhermore, we assume existence of a countable infinite set of variables $\var$ disjoint from $\sigma$ and, inductively, construct the set $\terms$ of all terms built using the signature $\signature$ and the variables set $\var$ in the following way: every variable is a $\terms$-term and if $f:n \in \signature$ and $t_1, \cdots, t_n \in \terms$ then $f(t_1, \dots, t_n) \in \terms$. The set of all variables occurring in a term $t$ is denoted by $\vars{t}$.

The structure of a term can be represented as a tree, where function symbols are nodes and arrows point to the arguments of the function.
\begin{figure}[!ht]
	\begin{displaymath}
		\xymatrix{
		&f|_\Lambda \ar@{-}^2[dr]\ar@{-}_1[dl]&\\
		a  & & g \ar@{-}^{21}[d] \\
		& &x
		}
	\end{displaymath}
	\caption{Tree representation of a term.}
	\label{figure:tree-representation-of-a-term}
\end{figure}
A precise formalism of describing subterm occurrences of a term is obtained by the notion of position. The set $\pos{t}$ of positions of $t$ is inductively defined as follows:
\begin{displaymath}
    \restr{t}{p} =
	\begin{cases}
		t                & \mbox{if } p = \Lambda                               \\
		\restr{t_i}{p_i} & \mbox{if } t = (t_1,\dots, t_n); \ \ 1 \leq i \leq n
	\end{cases}
\end{displaymath}

So positions are sequences of natural numbers denoting subterm occurrences. The set $\pos{t}$ is partitioned into two sets; $\basicPos{t}$, the set of all nonvariable positions of $t$, i.e., $\{p \in \pos{t} \mid \restr{t}{p} \notin \var \}$ and $\varPos{t}$, the set of all variable positions of $t$, $\{p \in \pos{t} \mid \restr{t}{p} \in \var \}$. If $p \in \pos{t}$ then $t[s]_p$ denotes the term obtained from $t$ by replacing the subterm of $t$ at position $p$ by $s$. Formally;
\begin{displaymath}
    t[s]_p =
	\begin{cases}
		s                & \mbox{if } p = \Lambda                               \\
		f(t_1, \dots, t_i[s]_q, \dots, t_n) & \mbox{if } t = f(t_1, \dots, t_n) \mbox{ and } p = i \cdot q
	\end{cases}
\end{displaymath}
Positions are partially ordered by the \textit{prefix order} $\leq$, i.e. $p \leq q$ if there exists an $r$ such that $q = p \cdot r$. We write $p < q$ for the strict part of $\leq$.

A substitution $\sigma$ over $\terms$ is a function $\sigma : \var \to \terms$ such that the set $\{ x \in \var \mid x \sigma \neq x \}$ is finite. The application of a substitution $sigma$ to a term $t$ is defined by indcution on the structure of terms:
\begin{displaymath}
    t\sigma =
	\begin{cases}
		x\sigma                & \mbox{if } t = x                               \\
		f(t_1\sigma, \dots, t_n\sigma) & \mbox{if } t = f(t_1, \dots, t_n)
	\end{cases}
\end{displaymath}
The domain of a substitution $\sigma$ is the set of variables that is not mapped to themselfs by $\sigma$. That is;
$$ \dom{\sigma} = \{x \in \var \mid x \sigma \neq x\}$$
the \textit{range} of $\sigma$ is the set of terms
$$ \ran{\sigma} = \bigcup\limits_{x \in \dom{\sigma}}  \{ x\sigma \},$$
and the set of variables occurring in the range is $\vran{\sigma} := \vars{\ran{\sigma}}$.

We represent substitutions in a more suitable way, that is, as a list of bidings from variables in the domain to terms in the range:
$$[x_1 / t_1, \dots, x_n / t_n ]$$
The restriction of a substitution $\sigma$ to a set of variables $V$, denoted by $\restr{\sigma}{V}$, is the substitution wich is equal to the identity everywhere except over $V \cap \dom{\sigma}$, where it is equal to $\sigma$. Composition of two substitutions is written, $\sigma \theta$, and is defined by
$$t\sigma \theta = (t\sigma)\theta.$$
An algorithm for constructing the composition of two substitution represented as sets of bidings is as follows:
\begin{enumerate}
    \item Apply $\theta$ to every term in $\ran{\sigma}$ to obtain $\sigma_1$;
    \item Remove from $\theta$ any biding $x/t$, where $x \in \dom{\sigma}$, to obtain $\theta_1$;
    \item Remove from $\sigma_1$ any trivial binding $x/x$, to obtain $\sigma_2$; and
    \item Take the union of the two list of bidings $\sigma_2$ and $\theta_1$.
\end{enumerate}
Two substitutions $\theta, \sigma$ are equal on the set of variables $V$, denoted by $\sigma =^V \theta$ iff $x \sigma = x \theta$, for all $x \in V$. A variable renaming is a bijective substitution on $\var$. A substitution $\sigma$ is \textit{more general} than a substitution $\tau$ on the set of variables $V$, denoted as $\sigma \leq^V \tau$, if there exists a substitution $\eta$ such that $ \tau =^V \sigma \eta$.


Two terms $s$ and $t$ are unifiable if there exists a substitution $\rho$, a so-called unifier of $s$ and $t$, such that $s \rho = t \rho$.
We save the following result for later use.

\begin{lemma}\label{lemma:unifiers-preserve-variables}
    If $\sigma$ is an idempotent most general unifier of two terms $s, t$ that have no variables in common then $\dom{\sigma} \cup \vran{\sigma} = \vars{s} \cup \vars{t}$.
\end{lemma}

Let $\sim$ be a binary relation on terms. We say that $\sim$ is closed under contexts if $s \sim t$ implies $u[s]_p \sim u[t]_p$, for all terms $u$ and positions $p \in \pos{u}$. The relation $\sim$ is closed under substitutions if $s\sigma \sim t\sigma$ whenever $s \sim t$, for all substitutions $\sigma$. A relation that is closed under contexts and substitutions is called a \textit{rewrite relation}.

An equation is an unordered pair $(s,t)$ of terms, written as $s = t$. Let $E$ be a set of equations. The rewrite relation $\contr_E$ generated by $E$ is defined as:
\begin{center}
    $s \contr_E t$ iff there exists a position $p \in \pos{s}$, a substitution $\sigma$ and a equation $l = r \in E$ such that \newline
    $\restr{s}{p} = l \sigma$ and $t = s[r\sigma]_p$.
\end{center}

The smallest symmetric relation that contains $E$ and is closed under contexts and substitutions is denoted by $\leftrightarrow_E$. The transitive-reflexive closure of $\leftrightarrow_E$ is denoted by $\eqUnif{E}$. This relations also extend to substitutions in the obvious way. We call $E$ the equational theory generated by the set of equational axioms $E$.
